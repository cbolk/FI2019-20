\begin{itemize}
\item sequenza di caratteri delimitata dal terminatore '\texttt{\textbackslash 0}'
\item allocazione dell'array con \texttt{N+1} elementi
\item acquisizione    \begin{itemize}
	\item \texttt{scanf}: segnaposto \texttt{\%s} e niente \texttt{\&} davanti al nome dell'array
	\item \texttt{gets}: serve poi solo il nome dell'array, trattandosi \textit{sempre} di caratteri
    \end{itemize}
\item scansione con condizione \texttt{s[i] != '\textbackslash 0'}
\end{itemize}
\mysep{}

\subsection{Stringa allo specchio}\stepcounter{numex}
Scrivere un programma che acquisisce una stringa di al pi\`u 25 caratteri, inverte l'ordine di tutti i caratteri in essa contenuta e la visualizza.

\begin{tags}
stringa. algoritmo.
\end{tags}

\getsol{srccode/stringaallospecchio.c}

\subsection{Stringa allo specchio senza vocali}\stepcounter{numex}
Scrivere un programma che acquisisce una stringa di al pi\`u 25 caratteri, inverte l'ordine di tutti i caratteri ed elimina tutte le vocali in essa contenute. Ci sono solo caratteri minuscoli.

\begin{tags}
stringa. algoritmo.\index{Programmi!Inverti stringa}
\end{tags}

\getsol{srccode/mirrornovoc.c}

versione alternativa

\getsol{srccode/mirrornovoc2.c}

\subsection{Anagrammi}\stepcounter{numex}
Scrivere un programma che acquisisce due stringhe di al pi\`u 20 caratteri e determina se una sia l'anagramma\index{Programmi!Anagramma} dell'altra, visualizzando 1 in caso affermativo, 0 altrimenti.
\begin{tags}
algoritmo. stringhe. dimensione dell'array.
\end{tags}

\getsol{srccode/anagramma.c}