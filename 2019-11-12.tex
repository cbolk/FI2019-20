\begin{itemize}
\item allocazione dinamica: utilit\`a
\item \texttt{malloc}
\item \texttt{free}
\item tipo \texttt{void *}
\end{itemize}

\subsection{Visualizza sequenza di numeri interi in ordine inverso}\stepcounter{numex}
Scrivere un programma che chiede all'utente la quantit\`a \texttt{n} di valori interi che intende inserire, quindi acquisisce \texttt{n} e li visualizza in ordine inverso.

\begin{tags}
allocazione dinamica\index{Allocazione dinamica}. \texttt{malloc}\index{malloc@\texttt{malloc}}. \texttt{free}\index{free@\texttt{free}}. \texttt{sizeof}\index{sizeof@\texttt{sizeof}}.
\end{tags}

\getsol{srccode/visualizzaseqintinversa.c}

\subsection{Stringa di vocali}\stepcounter{numex}
Scrivere un sottoprogramma che ricevuta in ingresso una stringa, crea una nuova stringa contenente tutte e sole le vocali contenute nella stringa di ingresso -- di dimensioni strettamente idonee a contenere tale stringa -- e la restituisce al chiamante.

\getsol{srccode/strvoc.c}


\subsection{Primi nel file}\stepcounter{numex}
Scrivere un sottoprogramma che ricevuto in ingresso il nome di un file restituisce al chiamante tutti i numeri primi in esso contenuti.

\begin{tags}
sottoprogramma. numero primo\index{Sottoprogrammi!Numero primo}.
\end{tags}

\getsol{srccode/getprimesfromfile.c}

versione in cui si usano due \texttt{return}\stepcounter{numex}

\getsol{srccode/getprimesfromfile_v2.c}

versione in cui si passano entrambi i parametri per indirizzo\stepcounter{numex}

\getsol{srccode/getprimesfromfile_v3.c}

versione in cui si crea una struttura dati e la si restituisce per valore\stepcounter{numex}

\getsol{srccode/getprimesfromfile_v4.c}

\prosep{}

\subsection{Anagramma -- Proposto}\stepcounter{numexp}
Scrivere un sottoprogramma che riceve in ingresso due stringhe (senz'altro di lunghezza uguale) e
restituisce 1 se le stringhe sono una l'anagramma dell'altra, 0 altrimenti.

\begin{tags}
sottoprogramma. stringhe. numero primo\index{Sottoprogrammi!Anagramma}.
\end{tags}

\getsol{srccode/anagramma.sub.c}
