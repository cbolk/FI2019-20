\subsection{Prepara un cocktail}\stepcounter{numex}

Si vuole scrivere un programma che, dati gli ingredienti di un cocktail, ne definisca la ricetta e la gradazione alcolica.
Si assume di avere solamente ingredienti liquidi nella ricetta e che tali ingredienti non siano mai pi\`u di 10.
Si ricorda, inoltre, che nelle ricette dei cocktail le quantit\`a degli ingredienti sono definite in parti (che non necessariamente sono intere).

Definire una struttura dati che possa memorizzare le informazioni riguardanti un ingrediente di un cocktail, che sono:
\begin{itemize}
    \item nome dell'ingrediente (dalla lunghezza massima di 50 caratteri)
    \item gradazione alcolica dell'ingrediente in percentuale (che si assume intera)
    \item numero di parti nel cocktail
\end{itemize}

Scrivere un programma che:
\begin{itemize}
    \item prende in ingresso un intero $n$ che corrisponde al numero di ingredienti nel cocktail,
      se il numero \`e minore di 1 o maggiore di 10, l'inserimento deve essere ripetuto;
    \item chiede in input, per ogni ingrediente, il nome, la gradazione alcolica e il numero di parti;
    \item richiede quanti litri $l$ di cocktail si vogliono preparare;
    \item stampa, per ogni ingrediente, la quantit\`a necessaria per la preparazione di $l$ litri di cocktail;
    \item stampa la gradazione alcolica del cocktail
\end{itemize}

\begin{tags}
\texttt{struct}. \texttt{typedef}. Array monodimensionali.
\end{tags}
\getsol{srccode/cocktail.c}


\subsection{Controlla Sudoku}\stepcounter{numex}
Il Sudoku \`e un gioco di logica in cui lo scopo \`e quello di completare una matrice $9 \times 9$ inserendo numeri tra 1 e 9.
Il gioco termina quando la matrice \`e piena e rispetta i seguenti vincoli:
\begin{itemize}
    \item ogni riga contiene tutti i numeri da 1 a 9
    \item ogni colonna contiene tutti i numeri da 1 a 9
    \item ogni matrice $3 \times 3$ ottenuta raggruppando righe e colonne contigue contiene tutti i numeri da 1 a 9.
      In questo gruppo di matrici si includono:
      \begin{itemize}
          \item la matrice ottenuta prendendo le righe 0-2 delle colonne 0-2;
          \item la matrice ottenuta prendendo le righe 0-2 delle colonne 3-5;
          \item la matrice ottenuta prendendo le righe 0-2 delle colonne 6-8;
          \item la matrice ottenuta prendendo le righe 3-5 delle colonne 0-2;
          \item ...
      \end{itemize}
\end{itemize}

Scrivere un sottoprogramma per ogni vincolo che, data una matrice di interi $9 \times 9$
e tutti i parametri strettamente necessari, restituisce 1 se il vincolo \`e rispettato, 0 altrimenti.
Scrivere infine un sottoprogramma che, data una matrice di interi $9 \times 9$ e tutti i parametri strettamente necessari, 
restituisce 1 se tale matrice rappresenta una soluzione valida per il gioco del Sudoku, 0 altrimenti.

\begin{tags}
Sottoprogrammi. Array bidimensionali.
\end{tags}
\getsol{srccode/sudoku.c}


\subsection{Zoo}\stepcounter{numex}
Definire una tipo di dato che contenga informazioni sugli animali di uno zoo.
Il nuovo tipo deve contenere:
\begin{itemize}
    \item nome dell'animale, es: "Leone" (lunghezza massima 50 caratteri)
    \item specie dell'animale, es: "Panthera Leo" (lunghezza massima 50 caratteri)
    \item numero di esemplari adulti nello zoo
    \item numero di cuccioli nello zoo
\end{itemize}

Scrivere i seguenti sottoprogrammi:
\begin{itemize}
    \item dato in ingresso l'array di animali dello zoo e tutti gli altri parametri strettamente necessari, 
      restituisce il numero totale di animali nello zoo;
    \item dato in ingresso l'array di animali dello zoo e tutti gli altri parametri strettamente necessari, 
      restituisce la percentuale di cuccioli rispetto al totale degli animali nello zoo;
    \item dato in ingresso l'array di animali dello zoo, una stringa contenente il nome di una specie 
      e tutti gli altri parametri strettamente necessari, restituisce il numero totale di animali 
      di quella specie all'interno dello zoo. 
      Si consiglia di definire un sottoprogramma di supporto che, date in ingresso due stringe, restituisce 1
      se le due stringhe sono identiche, 0 altrimenti.
\end{itemize}

\begin{tags}
\texttt{struct}. \texttt{typedef}. Array monodimensionali. Stringhe. Sottoprogrammi.
\end{tags}
\getsol{srccode/zoo.c}
