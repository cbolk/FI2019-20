\subsection{ISBN-10}\stepcounter{numex}
Scrivere un sottoprogramma \texttt{checkISBN10} che riceve in ingresso una stringa che rappresenta un codice ISBN-10 (International Standard Book Number) di 10 cifre numeriche separate da \texttt{-}, utilizzato per identificare univocamente un volume prima del 2007 (dal 2007 in poi il numero \`e di 13 cifre). Il sottoprogramma restituisce 1 se il codice ISBN \`e valido, 0 altrimenti.
Un codice ISBN-10 \`e valido se la somma delle somme \`e un multiplo di 11. La somma delle somme si calcola addizionando ogni cifra del codice alla somma delle precedenti cifre.
\\ \\
Esempi:
\\
Ingresso: \texttt{0-306-40615-2}
Cifre: \texttt{0 3 0 6 4 0 6 1 5 2}
Somma delle cifre: \texttt{0 3 3 9 13 13 19 20 25 27}
Somma delle somme: \texttt{132}
Uscita: \texttt{1}
\\
Ingresso: \texttt{0-07-881809-4}
Cifre: \texttt{0 0 7 8 8 1 8 0 9 4}
Somma delle cifre: \texttt{0 0 7 15 23 24 32 32 41 45}
Somma delle somme: \texttt{219}
Uscita: \texttt{0}
\\ \\
Scrivere il programma che acquisisce da riga di comando la stringa dell’ISBN-10 e - utilizzando il sottoprogramma checkISBN10 - visualizza 1 se il codice \`e corretto, 0 altrimenti.

\begin{esame}
26/01/2018
\end{esame}

\begin{tags}
stringhe. sottoprogrammi. argomenti da riga di comando.
\end{tags}

\getsol{srccode/isbn10.sub.c}


\subsection{Elimina le sottosequenze}\stepcounter{numex}
Definire un tipo di dato opportuno \texttt{clist\_t} per realizzare una lista dinamica che gestisce caratteri (ogni elemento un carattere) e serve per gestire sequenze di caratteri. Si definisce sottosequenza una sequenza di caratteri compresa tra una parentesi tonda iniziale \texttt{(} e una finale \texttt{)}.
Le sottosequenze possono anche essere vuote (parentesi aperta e poi chiusa senza altri caratteri intermedi). 
Scrivere un sottoprogramma riceve in ingresso una lista che costituisce una sequenza e la restituisce dopo aver sostituito le sottosequenze con il carattere \texttt{\#}.

La sequenza dei caratteri in ingresso \`e ben formata, ossia:
\begin{itemize}
    \item per ogni parentesi tonda che si apre, c'\`e una parentesi tonda che si chiude
    \item non ci sono intersezioni tra coppie di parentesi.
\end{itemize}
Per esempio, \texttt{ab(acg)be()a(xx)f} \`e una sequenza ben formata, \texttt{ab(a(c)g)b} e \texttt{aba(c)g)b} non lo sono.
Se il sottoprogramma riceve in ingresso la sequenza \texttt{ab(acg)be()a(xx)f(a)}, restituisce la sequenza \texttt{ab(\#)be(\#)a(\#)f(\#)}.
\\
Si considerino gi\`a disponibili e non da sviluppare i sottoprogrammi seguenti:
\\
\begin{lstlisting}[language=c]
/* inserisce in testa alla lista */
clist_t * push(clist_t *, char);

/* inserisce in coda alla lista */
clist_t * append( clist_t * , char );

/* elimina dalla lista il primo elemento */
clist_t * pop(clist_t *);

/* elimina dalla lista tutti gli elementi con il valore indicato */
clist_t * delete(clist_t *, char);

/* restituisce il riferimento all'elemento nella lista che ha il valore indicato, se esiste */
clist_t * exists(clist_t * , char );

/* restituisce il numero di elementi nella lista */
int length(clist_t *);
\end{lstlisting}

\begin{esame}
19/02/2019
\end{esame}

\begin{tags}
liste.
\end{tags}

\getsol{srccode/sottosequenze.sub.c}


\subsection{Conta le vette}\stepcounter{numex}
Scrivere un sottoprogramma ricorsivo che ricevuto in ingresso un array di interi (e qualsiasi altro parametro ritenuto strettamente necessario) restituisce il numero di \textit{vette} in esso presenti. Si definisce \textit{vetta} un valore presente nell'array maggiore di tutti i valori successivi presenti nell'array. L'ultimo valore dell'array, per definizione, non \`e una \textit{vetta}.
Scrivere quindi un programma che acquisisce in ingresso un array di 10 interi e stampa il rispettivo numero di \textit{vette}.

\begin{esame}
18/02/2016
\end{esame}

\begin{tags}
ricorsione. sottoprogrammi. array.
\end{tags}

\getsol{srccode/vette.sub.c}


\subsection{Da intero a lista}\stepcounter{numex}
Scrivere un sottoprogramma \texttt{int2list} che ricevuto in ingresso un numero intero restituisce una lista in cui ogni cifra del numero in ingresso compare tante volte quanto il suo valore. 
Nel caso in cui il valore ricevuto in ingresso sia negativo, il sottoprogramma restituisce la lista creata a partire dalla cifre in ordine opposto.

Se per esempio il sottoprogramma riceve in ingresso l'intero 3204, il sottoprogramma restituisce la lista seguente:
$3 \rightarrow 3 \rightarrow 3 \rightarrow 2 \rightarrow 2 \rightarrow 4 \rightarrow 4 \rightarrow 4 \rightarrow 4 \rightarrow |$


Se per esempio il sottoprogramma riceve in ingresso l'intero -3204, il sottoprogramma restituisce la lista seguente:
$4 \rightarrow 4 \rightarrow 4 \rightarrow 4 \rightarrow 2 \rightarrow 2 \rightarrow 3 \rightarrow 3 \rightarrow 3 \rightarrow |$


Definire inoltre un tipo di dato opportuno per la lista.


\begin{esame}
18/02/2019
\end{esame}

\begin{tags}
liste. allocazione dinamica.
\end{tags}

\getsol{srccode/inttolist.sub.c}


\subsection{Cifra pi\`u frequente}\stepcounter{numex}
Scrivere un sottoprogramma che riceve in ingresso un riferimento ad un file (gi\`a aperto) e legge un valore intero (se c'\`e ...) e restituisce la cifra del valore letto che in esso compare pi\`u di frequente. Nel caso in cui non ci sia un valore, restituisce -1. 
Nel caso ci siano pi\`u cifre che compaiono lo stesso numero di volte, restituisce quella pi\`u alta.
Se per esempio legge il valore 217319 restituisce 1, se legge il valore 1002932 restituisce 2.

\begin{esame}
28/01/2019
\end{esame}

\begin{tags}
file. sottoprogrammi.
\end{tags}

\getsol{srccode/cifrafrequente.sub.c}

