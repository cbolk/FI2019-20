\begin{itemize}
\item array pluri-dimensionali: dimensione 2
\begin{itemize}
\item scansione mediante due cicli annidati
\item linearizzazione della memoria
\end{itemize}
\item \texttt{struct}   
	\begin{itemize}
	\item nome facoltativo, ma sempre utilizzato
	\item array di \texttt{struct}
	\item \texttt{struct} con campo array
    \end{itemize}
\item \texttt{typedef}
\end{itemize}

\mysep{}

%\subsection{Palindrome}
%Scrivere un programma che acquisita una sequenza di 15 caratteri visualizza 1 se si tratta di un vocabolo palindrome\index{Programmi!Palindrome}, 0 altrimenti.
%
%\begin{tags}
%algoritmo. variabile di controllo che rappresenta il risultato.
%\end{tags}
%
%
%\getsol{srccode/palindrome.c}

\subsection{Matrice identit\`a}\stepcounter{numex}
Si scriva un programma che acquisisce i dati di una matrice di dimensione \texttt{5x5} di valori interi. Il programma visualizza 1 se si tratta di una matrice identit\`a, 0 altrimenti.

\begin{tags}
array bidimensionali. algoritmo.\index{Programmi!Matrice identit\`a}
\end{tags}

\getsol{srccode/matidentita.c}

\noindent\makebox[\linewidth]{\rule{\textwidth}{0.6pt}}

\subsection{Date: definizione di tipo}\stepcounter{numex}
Definire un nuovo tipo di dato, cui dare nome \texttt{date\_t} , per rappresentare date in termini di giorno, mese ed anno.
\begin{tags}
\texttt{typedef}\index{typedef@\texttt{typedef}}.
\end{tags}

\begin{lstlisting}[language=c]
typedef struct date_s {
   int g, m, a;
} date_t;
\end{lstlisting}

\subsection{Studente: definizione di tipo}\stepcounter{numex}
Definire un nuovo tipo di dato, cui dare nome \texttt{student\_t} per rappresentare le informazioni seguenti relative ad uno studente:
\begin{itemize}
\item cognome e nome: due array di caratteri di 41 caratteri ciascuno
\item data di nascita e data di immatricolazione: due data
\item voti esami: un array di NESAMI interi (dovete saperlo voi)
\item media: voto medio (relativo agli esami superati)
\item livello: 'T' o 'M' per distinguere se si \`e immatricolati alla triennale o alla magistrale. 
\end{itemize}

\begin{tags}
\texttt{typedef}\index{typedef@\texttt{typedef}}.
\end{tags}

\begin{lstlisting}[language=c]
#define NL 41
#define NESAMI 18
typedef struct student_s {
   char last[NL+1], first[NL]+1;
   date_t dnascita, dlaurea;
   int grades[NEX];
   float avg;
   char level;
} student_t;
\end{lstlisting}


