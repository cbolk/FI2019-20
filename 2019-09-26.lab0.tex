\begin{labex}{Conversione del tempo in ore, minuti e secondi}\stepcounter{numex}
Scrivere un programma che acquisito un valore intero positivo che rappresenta una durata temporale espressa in secondi, calcoli e visualizzi lo stesso intervallo di tempo espresso in ore, minuti e secondi.

\begin{tags}
algoritmo
\end{tags}

\begin{labexinout}
\labexin{un numero intero}
\labexout{tre interi separati da uno spazio (seguito da un carattere \texttt{'a-capo'})}
\end{labexinout}

\begin{labexcases}
\labexcin{453}
\labexcout{0 7 33}

\labexcin{43268}
\labexcout{12 1 8}

\end{labexcases}


\getsol{srccode/tempoinsec.c}

\end{labex}


\begin{labex}{Resto}\stepcounter{numex}
Scrivere un programma che acquisisce un intero e calcola e visualizza il numero di monete da 2 euro, 1 euro, 50, 20, 10, 5, 2 e 1 centesimo.

\begin{tags}
algoritmo
\end{tags}

\begin{labexinout}
\labexin{un numero intero}
\labexout{otto numeri interi}
\end{labexinout}

\begin{labexcases}
\labexcin{453}
\labexcout{2 0 1 0 0 0 1 1}

\labexcin{188}
\labexcout{0 1 1 1 1 1 1 1}

\end{labexcases}


\getsol{srccode/resto.c}

\end{labex}
