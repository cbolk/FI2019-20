\begin{labex}{Fattori}
Scrivere un programma che acquisisce due numeri interi relativi e visualizza 1 se uno \`e un divisore dell'altro o viceversa, 0 altrimenti.
Dopo il valore visualizzato, mettere un \texttt{'a-capo'}.

\begin{labexinout}
\labexin{due numeri interi}
\labexout{un intero (seguito da un carattere \texttt{'a-capo'})}
\end{labexinout}

\begin{labexcases}
\labexcin{5542 18}
\labexcout{0}

\labexcin{5542 17}
\labexcout{1}

\labexcin{13 1950}
\labexcout{1}
\end{labexcases}

\getsol{srccode/fattori.c}

\end{labex}


\begin{labex}{Padding}
Si vuole rappresentare a video un valore naturale \texttt{num} utilizzando un numero a scelta di cifre \texttt{k} inserendo 0 nelle posizioni pi\`u significative, fino a raggiungere la dimensione desiderata. 
Per esempio, volendo rappresentare \texttt{842} su \texttt{5} cifre, si ottiene \texttt{00842}.

Scrivere un programma che acquisisce due valori interi entrambi strettamente positivi (e finch\'e non \`e cos\`i richiede il valore che non rispetta il vincolo) num e \texttt{k}, quindi rappresenta \texttt{num} su \texttt{k} cifre. Se \texttt{k} e' minore del numero di cifre presenti in \texttt{num}, il programma visualizza il valore \texttt{num} come \`e. 
Dopo il valore visualizzato, mettere un \texttt{'a-capo'}.

%\begin{tags}
%\texttt{if}\index{if@\texttt{if}}
%\end{tags}

\begin{labexinout}
\labexin{due numeri interi (da verificare)}
\labexout{un intero (seguito da un carattere \texttt{'a-capo'})}
\end{labexinout}

\begin{labexcases}
\labexcin{11304 9}
\labexcout{000011304}

\labexcin{-4 9000 -5 -2 2}
\labexcout{9000}

\labexcin{1 1}
\labexcout{1}
\end{labexcases}


\getsol{srccode/padding.c}

\end{labex}

\begin{labex}{Super Mario}

\begin{wrapfigure}{r}{0.25\textwidth}
  \begin{center}\vspace{-2em}
    \includegraphics[width=\linewidth]{srccode/pyramids-mario.png}
  \end{center}\vspace{-8em}
\end{wrapfigure}



Nella preistoria dei videogiochi in Super Mario della Nintendo, Mario deve saltare da una piramide di blocchi a quella adiacente.
Proviamo a ricreare le stesse piramidi in C, in testo, utilizzando il carattere cancelletto ($\#$) come blocco, come riportato di seguito. In realt\`a il carattere $\#$ \`e pi\`u alto che largo, quindi le piramidi saranno un po' pi\`u alte.

\begin{verbatim}
   #  #
  ##  ##
 ###  ###
####  ####
\end{verbatim}

Notate che lo spazio tra le due piramidi \`e sempre costituito da \textbf{2} spazi, indipendentemente dall'altezza delle piramidi. Inoltre, alla fine delle piramidi \textbf{non ci devono essere spazi}.
L'utente inserisce l'altezza delle piramidi, che deve essere un valore strettamente positivo e non superiore a 16. In caso l'utente inserisca un valore che non rispetta questi vincoli, la richiesta viene ripetuta.

\begin{labexinout}
\labexin{un numero intero (da verificare)}
\labexout{una sequenza di caratteri}
\end{labexinout}


\getsol{srccode/mario.c}

\end{labex}

\begin{labex}{Scorrimento a destra -- rightshift}
Scrivere un programma che acquisita una stringa di al pi\`u 10 caratteri, modifica la stringa in modo tale che la stringa finale sia quella iniziale, fatta scorrere a destra di una posizione, con l'ultimo carattere riportato in testa. Se per esempio la sequenza iniziale \`e \texttt{attraverso}, la stringa finale sar\`a \texttt{oattravers}. 
Una volta modificata la stringa memorizzata, visualizzarla e farla seguire da un carattere \texttt{'a-capo'}.

\begin{labexinout}
\labexin{al pi\`u 10 caratteri}
\labexout{al pi\`u 10 caratteri (seguiti da un carattere \texttt{'a-capo'})}
\end{labexinout}

\begin{labexcases}
\labexcin{attraverso}
\labexcout{oattravers}

\labexcin{ananas}
\labexcout{sanana}

\labexcin{trampolini}
\labexcout{itrampolin}
\end{labexcases}

\getsol{srccode/rightshift.c}

\end{labex}


\begin{labex}{Troncabile primo a destra}
Scrivere un programma che acquisisce un valore intero strettamente positivo, e finch\'e non \`e tale lo richiede. Il programma analizza il valore intero e visualizza 1 nel caso sia un troncabile primo a destra, 0 altrimenti.
Un numero si dice troncabile primo a destra se il numero stesso e tutti i numeri che si ottengono eliminando una alla volta la cifra meno significativa del numero analizzato al passo precedente, sono numeri primi.
Per esempio, se il numero iniziale \`e 719, i numeri che si ottengono ``eliminando una alla volta la cifra meno significativa del numero analizzato al passo precedente ..'' sono 71 e 7.
Dopo il valore visualizzato, mettere un \texttt{'a-capo'}.

\begin{labexinout}
\labexin{un intero (da verificare)}
\labexout{un intero (seguito da un carattere \texttt{'a-capo'})}
\end{labexinout}

\begin{labexcases}
\labexcin{719}
\labexcout{1}

\labexcin{473}
\labexcout{0}

\labexcin{-42 -18 311111}
\labexcout{0}

\labexcin{3137}
\labexcout{1}
\end{labexcases}

\getsol{srccode/troncabileprimodx.c}

\end{labex}

