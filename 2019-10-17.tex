\subsection{Determinante di una matrice dimensione 3}\stepcounter{numex}

Scrivere un programma che acquisisce i dati di una matrice di dimensione \texttt{3x3} di numeri interi, quindi calcola e visualizzare il determinante.
\[
A = 
\begin{bmatrix}
a_{11} & a_{12} & a_{13} \\
a_{21} & a_{22} & a_{23} \\
a_{31} & a_{32} & a_{33} \\
\end{bmatrix}
\]

{\small
$$ det(A) = a_{11}\times a_{22}  \times a_{33}  + a_{12} \times a_{23}\times a_{31}  + a_{13} \times a_{21} \times a_{32} - a_{13} \times a_{22} \times a_{31} - a_{12} \times a_{21} \times a_{33} - a_{11} \times a_{23} \times a_{32} $$}

\begin{tags}
array bidimensionale.
\end{tags}

%\lstinputlisting[language=c]{srccode/determinante3.c}

\subsection{Nome di file da percorso, nome ed estensione}\stepcounter{numex}
Scrivere un programma che acquisisce una stringa di al pi\`u 50 caratteri che contenga il nome di un file completo di percorso e visualizza il solo nome. 

\begin{tags}
stringa. algoritmo.
\end{tags}

\begin{esame}
\end{esame}

\getsol{srccode/nomefile.c}

oppure:

\getsol{srccode/nomefile2.c}


\subsection{Da intero a stringa\index{Programmi!Da intero a stringa}}\stepcounter{numex}
Scrivere un programma che acquisito un intero crea una stringa che contiene le cifre dell'intero. L'intero ha al pi\`u 6 cifre. 

\begin{tags}
stringa. algoritmo. carattere numerico.
\end{tags}

\getsol{srccode/itos.c}

Versione alternativa.

\getsol{srccode/itos2.c}

\subsection{Area di un poligono}\stepcounter{numex}
Scrivere un programma che calcola e visualizza l'area di un poligono, a partire dalle coordinate dei suoi vertici e usando la formula di Gauss.
Il poligono ha al pi\`u 10 vertici, ciascuno rappresentato dalle coordinata \texttt{x} e \texttt{y} (valori interi).
Il programma acquisisce prima il numero \texttt{num} di vertici del poligono, verificando che sia non superiore a 10, quindi acquisisce le coordinate dei \texttt{num} vertici, quindi procede al calcolo dell'area e la visualizza.
Si dichiari un opportuno tipo di dato (\texttt{vertex\_t}) per rappresentare i vertici del poligono.

\getsol{srccode/areapoligonogauss.c}


\subsection{Quadrato magico}\stepcounter{numex}
Si scriva un programma che acquisisce i dati di una matrice di dimensione \texttt{3x3} di valori interi. Il programma visualizza 1 se si tratta di un quadrato magico seguito dal valore della somma, 0 altrimenti.
Un quadrato magico \`e un insieme di numeri interi distinti in una tabella quadrata tale che la somma dei numeri presenti in ogni riga, in ogni colonna e in entrambe le diagonali dia sempre lo stesso valore.

\begin{tags}
array bidimensionali. algoritmo.
\end{tags}

\getsol{srccode/magicq1.c}

Versione alternativa.

\getsol{srccode/magicq2.c}

