\subsection{Conta vocaboli}\stepcounter{numex}
Scrivere un programma che chiede all'utente il nome di un file (al pi\`u 40 caratteri) di testo e conta e visualizza il numero di parole presenti nel file. Le parole sono separate esclusivamente da spazi e sono al pi\`u di 32 caratteri.

\begin{tags}
file ASCII. Accesso a file con contenuto di lunghezza non nota a priori.\index{File!Accesso con numero di dati non noti a priori}
\end{tags}
\getsol{srccode/contavocabolifile.c}

Quest'altra soluzione non funziona: si finisce in un ciclo infinito. Per quanto il file sia finito, l'istruzione \texttt{fscanf} non cessa di essere eseguita e legge sempre l'ultimo vocabolo, restituendo 1.
\getsol{srccode/contavocabolifile_no.c}

\subsection{Copia file lowercase}\stepcounter{numex}
Scrivere un programma che chiede all'utente il nome di un file (al pi\`u 40 caratteri) di testo e scrive un nuovo file di testo, il cui nome \`e uguale a quello inserito con l'aggiunta del suffisso \texttt{"\_lowercase"}.
Il contenuto del nuovo file deve essere identico a quello del file originale, ma con le lettere maiuscole trasformate in minuscole.

\begin{tags}
file ASCII. Accesso a file con contenuto di lunghezza non nota a priori.\index{File!Accesso con numero di dati non noti a priori}
\end{tags}
\getsol{srccode/filelowercase.c}

\subsection{Record di un Videogame}\stepcounter{numex}
Abbiamo quasi terminato di implementare il nostro nuovo videogame, manca solo la possibilit\`a di memorizzare i migliori record.
Il file del salvataggio, che sar\`a in formato binario, conterr\`a (in successione) il numero di record che sono salvati all'interno del file e l'elenco dei record.

Come ogni videogioco che si rispetti, a ogni record \`e associato il nome del suo autore, di esattamente 3 lettere, e il rispettivo punteggio. 
Inoltre \`e possibile memorizzare al pi\`u i 10 record migliori.

Definire un nuovo tipo di dato che possa contenere le informazioni riguardanti un record.
Scrivere quindi i sottoprogrammi \texttt{load} e \texttt{save} che, dati il nome del file, l'array dei migliori record e tutti i parametri strettamente necessari:
\begin{itemize}
    \item legga dal file indicato i record (\texttt{load})
    \item scriva i record nel file indicato (\texttt{save})
\end{itemize}

Facoltativo:
Scrivere un sottoprogramma \texttt{insert} che, dato l'array dei migliori record, un nuovo record e tutti i parametri strettamente necessari, inserisca il nuovo record tra i migliori, se necessario.

\begin{tags}
file binario. \texttt{typedef}. \texttt{sizeof}.
\end{tags}

\getsol{srccode/videogame.c}

%\subsection{Classifica iniziali di parole}\stepcounter{numex}
%Scrivere un programma che chiede all'utente il nome di un file (al pi\`u 40 caratteri) di testo e conta e visualizza il numero di parole che iniziano con ciascuna lettera dell'alfabeto. I caratteri contenuti nel file sono solo minuscoli. Le parole sono separate esclusivamente da spazi e sono al pi\`u di 32 caratteri. 
%Visualizzare il messaggio: 
%\begin{verbatim}
%parole che iniziano con a: 10
%parole che iniziano con b: 3
%parole che iniziano con e: 14
%\end{verbatim}
%Non visualizzare alcun messaggio per le lettere che non hanno parole nel testo.
%
%\begin{tags}
%file ASCII. Accesso a file con contenuto di lunghezza non nota a priori. Contatori.
%\end{tags}
%\getsol{srccode/containizialivocabolifile.c}
%
%\subsection{Date di Nascita}\stepcounter{numex}
%
%Scrivere un programma che legge dal file binario \texttt{date.bin} le date di nascita di 100 persone, memorizzate come tre interi che rappresentano il giorno, il mese e l'anno di nascita. Il programma chiede all'utente una data e conta e visualizza i) il numero di persone nate in quel giorno, e ii) il numero di persone che festeggiano insieme il compleanno in tale data, in base ai dati contenuti nel file \texttt{date.bin}. Il programma visualizza i due valori interi, separati da uno spazio e seguiti da un carattere a-capo '\textbackslash n'. Definire un tipo opportuno per rappresentare le date.
%
%\begin{tags}
%file binario. \texttt{typedef}. \texttt{sizeof}.
%\end{tags}
%
%\getsol{srccode/datenascita.c}

