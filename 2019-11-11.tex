\subsection{Conta vocaboli}
Scrivere un programma che chiede all'utente il nome di un file (al pi\`u 40 caratteri) di testo e conta e visualizza il numero di parole presenti nel file. Le parole sono separate esclusivamente da spazi e sono al pi\`u di 32 caratteri.

\begin{tags}
file ASCII. Accesso a file con contenuto di lunghezza non nota a priori.
\end{tags}
\getsol{srccode/contavocabolifile.c}

Quest'altra soluzione non funziona: si finisce in un ciclo infinito. Per quanto il file sia finito, l'istruzione \texttt{fscanf} non cessa di essere eseguita e legge sempre l'ultimo vocabolo, restituendo 1.
\getsol{srccode/contavocabolifile_no.c}


\subsection{Quale lettera usare per un lipogramma}
Scrivere un programma che chiede all'utente il nome di un file (al pi\`u 40 caratteri) di testo e conta e visualizza il numero di parole che iniziano con ciascuna lettera dell'alfabeto. I caratteri contenuti nel file sono solo minuscoli. Le parole sono separate esclusivamente da spazi e sono al pi\`u di 32 caratteri. 
Visualizzare il messaggio: 
\begin{verbatim}
parole che iniziano con a: 10
parole che iniziano con b: 3
parole che iniziano con e: 14
\end{verbatim}
Non visualizzare alcun messaggio per le lettere che non hanno parole nel testo.

\begin{tags}
file ASCII. Accesso a file con contenuto di lunghezza non nota a priori. Contatori
\end{tags}
\getsol{srccode/containizialivocabolifile.c}
