\begin{itemize}
\item struttura di un programma
    \begin{itemize}
    \item dichiarazione di variabili
    \item elaborazione
    \item visualizzazione dei risultati
    \end{itemize}
\item istruzioni
\item commenti \texttt{/* */}
\item tipi di dati di base: \texttt{int}, \texttt{float}, \texttt{double}, \texttt{char}
\item \texttt{return 0}
\item \texttt{\#define}
\item input / output formattato
    \begin{itemize}
    \item acquisizione formattata \texttt{scanf}
    \item visualizzazione \texttt{printf}
    \end{itemize}
\end{itemize}

\mysep{}


\subsection{Tempo in ore, minuti e secondi}\stepcounter{numex}
Scrivere un programma che acquisito un valore intero che rappresenta un lasso di tempo espresso in secondi calcola e visualizza lo stesso tempo in ore, minuti e secondi.

\begin{tags}
assegnamento. commenti \texttt{/* */}, operatori.
\end{tags}

\getsol{srccode/tempoinsec.c}

Versione alternativa. Il numero di operazioni eseguite \`e lo stesso e le variabili risparmiate non fanno la differenza.

\getsol{srccode/tempoinsec.v2.c}
