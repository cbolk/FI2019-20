\begin{itemize}
\item costrutto ciclico \texttt{while}\index{Costrutti!\texttt{while}}
    \begin{itemize}
    \item valutazione dell'espressione tra parentesi
    \item esecuzione del corpo solo se l'espressione \`e vera
    \item l'istruzione che aggiorna l'espressione  \`e tipicamente l'ultima del corpo del ciclo
    \end{itemize}
\item costrutto ciclico \texttt{do-while}\index{Costrutti!\texttt{do-while}}
    \begin{itemize}
    \item valutazione dell'espressione tra parentesi
    \item esecuzione del corpo almeno una volta 
    \end{itemize}
\end{itemize}

\mysep{}

\subsection{Minimo, massimo e valor medio}\stepcounter{numex}
Scrivere un programma che acquisisce una sequenza di 53 valori interi e calcola e visualizza valor minimo, valor massimo e media dei valori.\\index{Programmi!Minimo, massimo e valor medio} 

\begin{tags}
costrutto \texttt{while}\index{while@\texttt{while}}
\end{tags}

\lstinputlisting[language=c]{srccode/minmaxmed53.c}

\subsection{Fattoriale}\stepcounter{numex}
Chiedere all'utente un valore non negativo, e fino a quando non \`e tale ripetere la richiesta, quindi calcolare e visualizzare il fattoriale\index{Programmi!Fattoriale}.

\begin{tags}
costrutto \texttt{while}\index{while@\texttt{while}}, costrutto \texttt{do-while}\index{do-while@\texttt{do-while}}.
\end{tags}

\lstinputlisting[language=c]{srccode/fattoriale.c}

versione alternativa che distingue il caso limite $num=0$ e $num=1$ inutilmente.

\lstinputlisting[language=c]{srccode/fattorialeno.c}

versione alternativa che distrugge inutilmente il valore iniziale senza un reale risparmio

\lstinputlisting[language=c]{srccode/fattorialeno2.c}

