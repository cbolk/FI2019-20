\subsection{Lunghezza di una stringa}\stepcounter{numex}
Scrivere un sottoprogramma che ricevute in ingresso due stringhe restituisce 1 se non hanno caratteri in comune, 0 altrimenti.

\getsol{srccode/strnocarcomune.c}

\subsection{Stringa nella stringa}\stepcounter{numex}
Scrivere un sottoprogramma che ricevute in ingresso due stringhe restituisce 1 se la seconda \`e una sottostringa della prima (\`e contenuta nella prima).

\getsol{srccode/strinstr.c}

\subsection{Cifre divisori}\stepcounter{numex}
Scrivere un sottoprogramma che ricevuto in ingresso un valore intero calcola e restituisce il numero di cifre del numero che sono divisori del numero stesso.

\getsol{srccode/cifrediv.c}

\subsection{Insieme unione ordinato}\stepcounter{numex}
Scrivere un sottoprogramma che ricevuti in ingresso due array di numeri interi e qualsiasi altro parametro ritenuto strettamente necessario, ordina entrambi gli array in senso crescente e \textit{visualizza} l'insieme unione ordinato.

\getsol{srccode/viewsetunionsorted.sub.c}


