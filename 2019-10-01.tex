\subsection{Conversione in binario, allo specchio}\stepcounter{numex}
Scrivere un programma che acquisito un valore intero positivo visualizza la sua rappresentazione in base 2 (con i bit in ordine inverso).

\begin{tags}
algoritmo. resto della divisione, \texttt{\#define}\index{define@\texttt{\#define}}.
\end{tags}

\getsol{srccode/dec2bin.rev.c}

\subsection{Potenza del 2}\stepcounter{numex}
Scrivere un programma che acquisito un valore \texttt{n} intero positivo calcola e stampa la prima potenza del 2 superiore ad \texttt{n}.

\begin{tags}
algoritmo. cicli, \texttt{while}\index{while@\texttt{while}}, \texttt{\#define}\index{define@\texttt{\#define}}.
\end{tags}

\getsol{srccode/minpow2.c}

\subsection{Numero di cifre di un valore}\stepcounter{numex}
Scrivere un programma che acquisito un valore intero calcola e visualizza il numero di cifre di cui \`e composto.\index{Programmi!Quantit\`a di cifre di un numero intero}

\getsol{srccode/sizenum.c}

\subsection{Einstein}\stepcounter{numex}
Si dice che Einstein si divertisse molto a stupire gli amici con il gioco qui riportato.

Scrivete il numero 1089 su un pezzo di carta, piegatelo e datelo ad un amico che lo metta da parte. Ci\`o che avete scritto non deve essere letto fino a che non termina il gioco.

A questo punto chiedere ad una persona di scrivere 3 cifre qualsiasi (ABC), specificando che la prima e l'ultima cifra devono differire di almeno due ($|A-C| >=2$). Per fare atmosfera, giratevi e chiudete gli occhi. Una volta scritto il numero di tre cifre, chiedete all'amico di scrivere il numero che si ottiene invertendo l'ordine delle cifre (CBA). A questo punto fate sottrarre dal numero pi\`u grande quello pi\`u piccolo: $XYZ = |ABC - CBA|$ e fate anche calcolare il numero che si ottiene invertendo le cifre del numero risultante: ZYX. Infine, fate sommare questi ultimi due numeri: XYZ + ZYX e constatate che il risultato \`e 1089.

Fate estrarre il foglio tenuto da parte fino a questo momento: 1089!

Realizzate un programma che fa i seguenti passi:\begin{itemize}
\item    Chiede all'utente di inserire un numero di 3 cifre, specificando il vincolo tra la prima e l'ultima cifra, e se i vincoli non sono rispettati chiede nuovamente il valore
\item    Acquisisce il valore
\item    Visualizza il numero inserito e il numero con le cifre in ordine inverso
\item    Visualizza la differenza tra i due valori (deve essere un numero positivo)
\item    Visualizza il numero con le cifre in ordine inverso
\item    Visualizza il risultato della somma tra questi due valori (dovrebbe essere 1089 per qualsiasi numero di 3 cifre che rispetta il vincolo tra la prima e l'ultima cifra)
\end{itemize}

\getsol{srccode/einstein.c}

\prosep{}

\subsection{Conversione in binario, senza array -- Proposto e risolto}\stepcounter{numex}
Acquisire un valore intero e calcolare e visualizzare la sua rappresentazione nel sistema binario, mostrando le cifre nell'ordine corretto. Se l'utente inserisce 6, il valore visualizzato \`e 110.
Il suggerimento \`e che un numero binario pu\`o anche essere visto come un numero in base dieci, ossia 110 in binario pu\`o anche essere visto come centodieci in decimale ...

\begin{tags}
algoritmo. resto della divisione, \texttt{\#define}\index{define@\texttt{\#define}}. moltiplicazioni ripetute per la base.
\end{tags}

\getsol{srccode/dec2bin.noarray.c}

\subsection{Conversione in binario, allo specchio, con dimensione}\stepcounter{numexp}
Scrivere un programma che acquisisce un primo valore intero, il dato da convertire in binario, ed un secondo dato intero, il numero di bit su cui rappresentarlo. Il programma visualizza la rappresentazione in base 2 (con i bit in ordine inverso) del valore acquisito, eventualmente aggiungendo bit di padding.

\getsol{srccode/dec2bin.dimrev.c}
