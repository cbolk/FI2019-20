\begin{itemize}
\item intestazione
\item tipo di dato restituito
\item parametri formali e attuali
\item prototipo 
\item \texttt{return}
\item tipo \texttt{void}
\item passaggio array monodimensionali e dimensione
\item visibilit\`a delle variabili
\end{itemize}

\subsection{Combinazioni\index{Programmi!Numero di combinazioni}}\stepcounter{numex}
Scrivere un programma che acquisisce due interi \texttt{n} e \texttt{k} strettamente positivi, con \texttt{k} non superiore a  \texttt{n} e finch\`e non \`e tale li richiede (entrambi). Quindi calcola e visualizza il numero di combinazioni di \texttt{n} elementi, presi \texttt{k} per volta.

\getsol{srccode/combinazioninosub.c}

versione con sottoprogrammi

\getsol{srccode/combinazioni.sub.c}

\getsol{srccode/combinazioni.c}

\subsection{Massimo di un array}\stepcounter{numex} 
Scrivere un sottoprogramma che ricevuto in ingresso un array di numeri interi e \textit{qualsiasi altro parametro ritenuto strettamente necessario} restituisce l'indice dell'elemento con valore massimo.

\begin{tags}
sottoprogrammi. passaggio array monodimensionale.
\end{tags}

\getsol{srccode/imaxarrayint.c}


\subsection{Triangolo}\stepcounter{numex} 
Scrivere un sottoprogramma che ricevuti in ingresso tre interi, restituisce 1 nel caso in cui si tratti delle dimensioni dei lati di un triangolo valido 0 altrimenti. Le condizioni che devono sussistere sono:\begin{itemize}
\item le dimensioni sono positive
\item la somma di due dimensioni non \`e mai superiore alla terza dimensione
\end{itemize}

\getsol{srccode/triangolovalido.c}

