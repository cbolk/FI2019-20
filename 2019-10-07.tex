\begin{itemize}
\item array monodimensionali
\item dati di tipo omogeneo
\item dimensione dell'array costante e nota a priori
\item indice degli elementi da \texttt{0} a \texttt{dimensione} - 1
\item costrutto ciclico a conteggio \texttt{for} (a condizione iniziale)\index{Costrutti!\texttt{for}}
\item le tre parti del costrutto
\begin{itemize}
	\item istruzioni prima di iniziare il ciclo
	\item condizioni, semplici e composte
	\item istruzioni a chiusura del corpo del ciclo
  \end{itemize}
\item separatore per le istruzioni nella prima e terza parte del costrutto
\item parti sono facoltative
\end{itemize}

\mysep{}


\subsection{5 dati in ingresso in ordine inverso}\stepcounter{numex}
Scrivere un programma che acquisiti 5 numeri interi li visualizza in ordine inverso.

\begin{tags}
array monodimensionale. cicli a conteggio. costrutto \texttt{for}\index{for@\texttt{for}}.
\end{tags}

\getsol{srccode/rev5noarray.c}

Versione con array.

\getsol{srccode/rev5.c}


\subsection{Sopra soglia}\stepcounter{numex}
Scrivere un programma che acquisisce 20 valori interi, quindi un intero valore soglia e calcola e visualizza in numero di campioni strettamente superiori alla soglia.

\getsol{srccode/soprasoglia.c}

Versione con il costrutto \texttt{for}.

\getsol{srccode/soprasoglia-for.c}

\prosep{}

\subsection{Conversione in binario -- Proposto}\stepcounter{numexp}
Scrivere un programma che acquisisce un valore compreso tra 0 e 1023, estremi inclusi e finch\'e non \`e tale lo richiede. Quindi effettua la conversione in base 2 e la visualizza.

\begin{tags}
algoritmo. validazione ingresso. dimensione dell'array.\index{Programmi!Da decimale a binario}
\end{tags}

\getsol{srccode/dec2bin.c}

Versione alternativa con ciclo \texttt{do-while}.

\getsol{srccode/dec2bin2.c}

Versione alternativa in cui si visualizza il numero sul numero completo di bit massimo.

\getsol{srccode/dec2binallbits.c}


