\begin{itemize}
\item variabili globali
\end{itemize}

\mysep{}

\subsection{Da array a lista istogramma}\stepcounter{numex}
Scrivere un sottoprogramma \texttt{array2istogramma} che riceve in ingresso un array di valori interi e qualsiasi altro parametro ritenuto strettamente necessario e restituisce una lista \textit{istogramma} che rappresenta un'unica volta tutti i valori presenti nell'array con il numero di occorrenze. La lista \`e ordinata in senso crescente rispetto ai valori (non rispetto al numero di occorrenze).

Scrivere poi un programma che acquisisce da file binario \texttt{dati.bin} 100 valori interi, chiama il sottoprogramma \texttt{array2istogramma} e visualizza un istogramma orizzontale utilizzando il carattere \texttt{*}, ordinato prima rispetto al valore, poi ordinato in senso crescente in base al numero di occorrenze.

Per esempio, se il file contiene i seguenti valori (20 per questioni di spazio):

\begin{verbatim}
4 5 1 2 6 8 9 1 7 2 4 5 6 1 4 8 2 0 3 11
\end{verbatim}

il programma visualizza:

\begin{verbatim}
0	: *
1	: ***
2	: ***
3	: *
4	: ***
5	: **
6	: **
7	: *
8	: **
9	: *
11	: *

0	: *
3	: *
7	: *
9	: *
11	: *
5	: **
6	: **
8	: **
1	: ***
2	: ***
4	: ***
\end{verbatim}

\begin{tags}
Lista concatenata semplice\index{Lista concatenata semplice}. Conta occorrenze\index{Programmi!Conta occorrenze}.
\texttt{increasing}\index{Lista concatenata semplice!\texttt{increasing}}.
\end{tags}

\getsol{srccode/arr2histo.c}

\subsection{Valori sotto la media}\stepcounter{numex}
Scrivere un sottoprogramma che riceve in ingresso un array di valori interi e qualsiasi altro parametro ritenuto strettamente necessario e restituisce una lista contenente tutti e soli i valori dell'array che sono minori o uguali alla media dei valori contenuti nell'array stesso.

\begin{tags}
ista concatenata semplice. algoritmo. calcolo della media dei valori.
\end{tags}

\begin{esame}
04/09/2015
\end{esame}

\getsol{srccode/listbelowavg.sub.c}

 
