\begin{labex}{Somma a k}
Scrivere un programma che acquisisce una sequenza di al pi\`u 100 valori interi e un intero strettamente positivo \texttt{k}. L'acquisizione della sequenza termina quando l'utente inserisce un numero negativo o nullo, oppure quando vengono acquisiti 100 valori. Il programma visualizza 1 se la sequenza contiene due valori tali che la loro somma sia \texttt{k}, 0 altrimenti. Dopo il valore visualizzato, mettere un \texttt{'a-capo'}.
Per realizzare la soluzione si sviluppi un sottoprogramma \texttt{cercasomma} che ricevuto in ingresso \texttt{k}, l'array contenente i dati e qualsiasi altro parametro ritenuto strettamente necessario, restituisce 1 o 0 nel caso trovi i due valori la cui somma \`e \texttt{k}.

\begin{labexinout}
\labexin{una sequenza di al pi\`u 101 valori interi}
\labexout{un intero (seguito da un carattere \texttt{'a-capo'})}
\end{labexinout}

\begin{labexcases}
\labexcin{10 15 3 7 -4 17}
\labexcout{1}

\labexcin{2 5 6 55 10 -11 100}
\labexcout{0}

\labexcin{2 2 3 -8 7}
\labexcout{0}
\end{labexcases}

\getsol{srccode/somma_a_k.c}

\end{labex}

\begin{labex}{Solo in ordine}
Scrivere un programma che acquisisce una sequenza di al pi\`u 20 valori interi, chiedendo all'utente inizialmente quanti valori vorr\`a fornire, \texttt{num}. 
Il programma acquisisce \texttt{num} valori e memorizza in una opportuna struttura dati la sequenza di valori i cui elementi sono strettamente crescenti, trascurando i valori che risultano non essere ordinati. Al termine dell'acquisizione il programma visualizza la lunghezza della sequenza, seguita, su una nuova riga, dalla sequenza stessa.
L'utente inserir\`a sempre un numero di valori coerente con la richiesta.
Avvalersi di due sottoprogrammi: \texttt{fillarrord} e \texttt{viewarr}: il primo memorizza i dati ritenuti validi, il secondo visualizza il contenuto di un array.

\begin{labexinout}
\labexin{sequenza di interi}
\labexout{sequenza di interi}
\end{labexinout}

\begin{labexcases}
\labexcin{10}
\labexcinextra{3 1 4 5 -1 3 1 4 5 -1}
\labexcout{3}
\labexcinextra{3 4 5}

\labexcin{6}
\labexcinextra{-1 3 -1 3 -1 3}
\labexcout{2}
\labexcinextra{-1 3}

\labexcin{8}
\labexcinextra{9 8 7 6 5 4 3 2}
\labexcout{1}
\labexcinextra{9}

\end{labexcases}

\getsol{srccode/filterup.c}

\end{labex}


\begin{labex}{Quadro di parole}
Scrivere un programma che acquisisce un valore intero strettamente positivo \texttt{num}, che rappresenta il numero di parole (ciascuna di al pi\`u 25 caratteri) che verranno poi fornite, e che comunque non saranno mai pi\`u di 20.
Il programma acquisisce le \texttt{num} parole e le visualizza, una per riga, all'interno di un rettangolo creato dal carattere \texttt{*}.

Per esempio, se l'utente fornisce:
\begin{verbatim}
5
Hello
world
in
un
rettangolo
\end{verbatim}

il programma visualizza:

\begin{verbatim}
************
*Hello     *
*world     *
*in        *
*un        *
*rettangolo*
************
\end{verbatim}

\begin{labexinout}
\labexin{un intero e una sequenza di stringhe}
\labexout{una sequenza di caratteri}
\end{labexinout}


\getsol{srccode/quadro.c}

\end{labex}


\begin{labex}{Mix di due array ordinati}
Scrivere un sottoprogramma che acquisisce due sequenze di valori interi, ciascuna di 20 elementi.
Il programma ordina le due sequenze in senso crescente, quindi visualizza la sequenza dei valori acquisiti, in senso crescente e senza ripetizioni. Al termine dell'esecuzione, le due sequenze sono ordinate.
Nel realizzare la soluzione, scrivere un sottoprogramma \texttt{sortarr} che ricevuto in ingresso un array, un intero \texttt{updown} e qualsiasi altro parametro ritenuto strettamente necessario, ordina il contenuto dell'array in senso crescente se \texttt{updown} vale 1, in senso decrescente se vale -1

\begin{labexinout}
\labexin{quaranta sequenze di numeri positivi}
\labexout{una sequenza di numeri positivi}
\end{labexinout}

\begin{labexcases}
\labexcin{-1 2 4 2 5 6 8 1 0 7 3 4 9 -9 9 9 9 9 9 9}
\labexcinextra{7 3 4 5 6 7 8 9 2 1 0 5 6 8 1 0 1 2 4 2}
\labexcout{-9 -1 0 1 2 3 4 5 6 7 8 9}

\labexcin{9 8 7 6 5 4 3 2 1 0 0 0 0 0 0 0 0 0 0 0}
\labexcin{1 2 2 2 2 2 2 2 2 2 2 2 2 2 2 2 2 2 2 2 }
\labexcout{0 1 2 3 4 5 6 7 8 9 }
\end{labexcases}

\getsol{srccode/sort2.c}

\end{labex}


\begin{labex}{Sottostringa pi\'u lunga senza ripetizioni}
Scrivere un programma che acquisita una stringa di al pi\'u 30 caratteri, individui la sottostringa pi\`u lunga in essa contenuta, senza caratteri ripetuti. Il programma visualizza la lunghezza di tale sottostringa, seguita da un carattere \texttt{'a-capo'}.

\begin{labexinout}
\labexin{una stringa}
\labexout{un intero}
\end{labexinout}

\begin{labexcases}
\labexcin{abcabcbb}
\labexcout{3}

\labexcin{alfabeto}
\labexcout{7}

\labexcin{bbbbb}
\labexcout{1}
\end{labexcases}

\getsol{srccode/sottostringa.c}

\end{labex}

