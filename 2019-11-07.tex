\begin{itemize}
\item utilit\`a dei file, differenze tra file ASCII e file binari.
\item accesso a file ASCII
    \begin{itemize}
    \item \texttt{FILE * fopen(char [], char []);}
    \item \texttt{int fclose(FILE *);}
    \item \texttt{int fscanf(FILE *, char [], ...);}
    \item \texttt{int fprintf(FILE *, char [], ...);}
    \item \texttt{char * fgets(char [], int, FILE *);}
    \item \texttt{int feof(FILE *);}
    \end{itemize}
\item lettura fino alla fine del file:
\begin{lstlisting}[language=c]
fscanf(fp, "%c", &car);
while(!feof(fp)){
	/* utilizzo del valore letto e memorizzato in car */
	/* ... */
	fscanf(fp, "%c", &car);
}
\end{lstlisting}
\item accesso a file binari
	\begin{itemize}
	\item \texttt{size\_t fread(void *ptr, size\_t size, size\_t nmemb, FILE *stream);}
	\item \texttt{size\_t fwrite(const void *ptr, size\_t size, size\_t nmemb, FILE *stream);}
	\end{itemize}
\end{itemize}


\subsection{Numeri pari e dispari}\stepcounter{numex}

Scrivere un programma che legge 100 valori interi dal file di testo ASCII \texttt{./dati.txt} e scrive in numeri pari nel file \texttt{./pari.txt} e quelli dispari nel file \texttt{./dispari.txt}. 

\getsol{srccode/fileparidispari.c}

\subsection{Numeri pari e dispari binari}\stepcounter{numex}

Scrivere un programma che legge 100 valori interi dal file binario \texttt{./dati.bin} e scrive in numeri pari nel file \texttt{./pari.txt} e quelli dispari nel file \texttt{./dispari.txt}. 

\getsol{srccode/fileparidisparibin.c}

\subsection{Numeri primi .. chiss\`a quanti}\stepcounter{numex}

Scrivere un programma che legge dal file di testo ASCII \texttt{./dati.txt} numeri interi (non si sa quanti dati il file contiene) e scrive i numeri primi nel file \texttt{primi.txt}. 

\getsol{srccode/fileprimi.c}
