\begin{itemize}
\item passo base
\item passo induttivo
\item condizione di termine della ricorsione
\item stack delle chiamate
\item visibilit\`a variabili locali
\item variabili \texttt{static}
\end{itemize}

\mysep{}

\subsection{Fattoriale -- versione ricorsiva}\stepcounter{numex}
Scrivere un sottoprogramma ricorsivo che calcola e restituisce il fattoriale\index{Sottoprogrammi!Fattoriale}\index{Sottoprogrammi ricorsivi!Fattoriale} di un intero.

\begin{tags}
sottoprogramma. ricorsione. 
\end{tags}

\getsol{srccode/fattoriale.subr.c}

\subsection{Lunghezza di una lista -- versione ricorsiva}\stepcounter{numex}
Scrivere un sottoprogramma ricorsivo che ricevuta in ingresso una lista, calcola e restituisce la sua lunghezza\index{Programmi lista concatenata semplice!Lunghezza di una lista}\index{Sottoprogrammi ricorsivi!Lunghezza di una lista}.

\begin{tags}
sottoprogramma. ricorsione\index{Ricorsione}. 
\texttt{length}\index{Lista concatenata semplice!\texttt{length}}
\end{tags}

\getsol{srccode/strlen.subr.c}

\subsection{Lunghezza di una stringa -- versione ricorsiva}\stepcounter{numex}
Scrivere un sottoprogramma ricorsivo che ricevuta in ingresso una stringa, calcola e restituisce la sua lunghezza\index{Sottoprogrammi!Lunghezza di una stringa}\index{Sottoprogrammi ricorsivi!Lunghezza di una stringa}.

\begin{tags}
sottoprogramma. ricorsione\index{Ricorsione}. stringhe.
\end{tags}

\getsol{srccode/strlen.subr.c}

\subsection{Carattere nella stringa -- versione ricorsiva}\stepcounter{numex}
Scrivere un sottoprogramma ricorsivo che ricevuta in ingresso una stringa ed un carattere, calcola e restituisce il numero di volte che il carattere compare nella stringa.

\begin{tags}
sottoprogramma. ricorsione\index{Ricorsione}. stringhe. indirizzo di un elemento dell'array.
\end{tags}

\getsol{srccode/charinstr.subr.c}

\subsection{Variabili \texttt{static}}
Programma che chiama un sottoprogramma con una variabile dichiarata \texttt{static}\index{Variabili \texttt{static}}, che viene allocata non sullo stack e il cui valore \`e persistente rispetto a chiamate successive.

\getsol{srccode/staticvar.c}

\subsection{Scompatta lista}\stepcounter{numex}
Scrivere un sottoprogramma \texttt{extract} che riceve in ingresso una lista per la gestione dei numeri interi, e un intero \texttt{start} che vale senz'altro o 0 o 1 (non \`e necessario gestire il caso in cui non sia cos\`i). La lista \textit{codifica} un'informazione binaria: il valore del primo elemento indica quante volte consecutive compare il bit \texttt{start}, il secondo elemento indica quante volte compare il complemento di \texttt{start}, il terzo elemento quante volte compare il bit  \texttt{start} e cos\`i fino alla fine.
Il sottoprogramma scompatta tale informazione e restituisce \textbf{una nuova lista} i cui elementi contengono ciascuno 0 o 1. 

Non \`e necessario definire due tipi diversi di dato, poich\`e il contenuto dell'elemento della lista \`e comunque sempre un intero. 

Per esempio, se la lista in ingresso \`e quella di seguito riportata e \texttt{start} \`e 1:
$$
3 \rightarrow 5 \rightarrow 1 \rightarrow 2  \rightarrow 2 \rightarrow 1 \rightarrow|
$$ 

il sottoprogramma restituisce la lista seguente

$$
1 \rightarrow 1 \rightarrow 1 \rightarrow 0 \rightarrow 0 \rightarrow 0 \rightarrow 0 \rightarrow 0 \rightarrow 1 \rightarrow 0 \rightarrow 0 \rightarrow 1 \rightarrow 1 \rightarrow 0 \rightarrow|
$$ 

\begin{tags}
lista concatenata semplice\index{Lista concatenata semplice}. 
\end{tags}

\begin{esame}
09/09/2019
\end{esame}


\getsol{srccode/20190909.liste.c}

\prosep{}

\subsection{Paint ... semplificato -- Proposto}
Si scriva un sottoprogramma \texttt{paint} che riceve in ingresso un array bidimesionale (con \texttt{NC} numero di colonne) e qualsiasi altro parametro ritenuto necessario, oltre a due interi \texttt{x} e \texttt{y} che sono le coordinate di un elemento dell'array. L'array contiene solo 0 e 1. Il sottoprogramma, modifica il valore dell'elemento di coordinate \texttt{x} e \texttt{y} mettendolo a 0, se vale 1, e propaga la trasformazione verso i 4 elementi adiacenti posti nelle posizioni a destra, sinistra, alto e in basso.

\begin{tags}
sottoprogramma. ricorsione. 
\end{tags}

