\begin{itemize}
\item allocazione dinamica: liste concatenate semplici
\item tipo di dato per elementi della lista
\begin{itemize}
	\item testa della lista
	\item elementi generici
	\item ultimo elemento della lista
\end{itemize}
\end{itemize}

\subsection{Duplica stringa}\stepcounter{numex}\stepcounter{numex}\stepcounter{numex}
Scrivere un sottoprogramma che ricevuta in ingresso una stringa ne crea e restituisce una con lo stesso contenuto.

\begin{tags}
allocazione dinamica\index{Allocazione dinamica}. \texttt{malloc}\index{malloc@\texttt{malloc}}. \texttt{sizeof}\index{sizeof@\texttt{sizeof}}. \texttt{strdup}\index{Stringhe!\texttt{strdup}}
\end{tags}

\getsol{srccode/strdup.sub.c}

\subsection{\texttt{itos}}\stepcounter{numex}
Scrivere un sottoprogramma che ricevuto in ingresso un valore intero positivo restituisce la stringa i cui caratteri sono le cifre del valore in ingresso.

\getsol{srccode/itostr.c}


\subsection{Tipo da lista}\stepcounter{numex}
Definire un tipo di dato opportuno per la realizzazione di una lista concatenata semplice per la gestione dei valori interi.

\getsol{srccode/list.type.c}
